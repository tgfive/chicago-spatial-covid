\begin{slide}{Basic Reproduction Number}
\vfill
The website \href{https://rt.live/}{Rt Live} tracked $R_0$ since the beginning of the COVID-19 pandemic using data from \href{https://covidtracking.com/?source=rt}{The COVID Tracking Project}.
\begin{columns}[T]
	\begin{column}{.5\textwidth}
		\begin{figure}[h]
			\centering
			\includegraphics[height=1.25in]{images/chicago_r_num}	
		\end{figure}
	\end{column}
	\begin{column}{.5\textwidth}
		\begin{block}{Model (\cite{rtlive2020})}
			Search for the most likely curve that produced the new cases per day that is observed:
			\begin{itemize}
				\item Assume a seed number of people and a $R_0$ curve,
				\item Distribute these cases into the future using a known delay distribution,
				\item Scale and add noise.
			\end{itemize}
		\end{block}
	\end{column}
\end{columns}
\vfill
Based the calculation for Illinois, we assume a logistic form for $R_0$ over time:
$$R_0 (t) = \frac{R_{\text{start}} - R_{\text{end}}}{1 + \exp ( - k (x_0 - t))} + R_{\text{end}}$$
\vfill
\end{slide}