\documentclass[11pt,letterpaper]{article}
\usepackage[utf8]{inputenc}
\usepackage[T1]{fontenc}
\usepackage{amsmath}
\usepackage{amsfonts}
\usepackage{amssymb}
\usepackage{graphicx}
\usepackage{ragged2e}

\pagenumbering{gobble}

%intsitute command
\usepackage{etoolbox}
\makeatletter
\providecommand{\institute}[1]{% add institute to \maketitle
	\apptocmd{\@author}{\end{tabular}
	\par
	\begin{tabular}[t]{c}
		\small \textit{#1}}{}{}
}
\makeatother

%margins
\usepackage[letterpaper, total={6.5in, 9.5in}]{geometry}

\begin{document}
	\vspace{1in}
	
	\begin{center}
		{\Large Response to Referee Requests for ``Reaction-diffusion spatial modeling of COVID-19 in Chicago''}
	\end{center}

	\vspace{0.25in}
		
	
	I would like to thank the anonymous reviewers of my project writeup, ``Reaction-diffusion spatial modeling of COVID-19 in Chicago'' for their comments and critiques.
	
	I've revised the paper to address multiple mathematical and form errors, including the presentation of the optimization problem as well as the vector notation.
	I have also included phrases to better introduce some of the mathematics.
	For example, the number of contacts and the diffusion coefficient are now introduced before the presentation of the dynamical system.
	The acronyms for the various epidemic models discussed are now spelled out, as requested.
	
	Additionally, I've introduced language to hopefully clarify the connection between my project and the values of the SoReMo initiative.
	In truth, I was hoping to obtain better mathematical results, in which case I had planned on including visualizations of inequalities in the city overlayed on the model results.
	I had also hoped to be able to offer suggestions for further use of the model to study how the pandemic interacted with social justice; unfortunately I was forced to offer thoughts on the failure of the model to reproduce the data.
	
	Lastly, in light of the overall failure of the model, I believe the request to include a discussion of the reasonableness of the parameters is beyond the scope of this project.
	Had the model been successful in describing the propagation of the disease, a discussion of the biological implication of the model would certainly be warranted.
	Since that did not happen, I'm not sure how much value such a discussion would add.
	I think the question of reasonableness is already addressed within the current scope of the project, in that the underlying assumptions about the transmission of the virus need to be revisited.
	
	\vspace{0.25in}
	
	\begin{FlushRight}
		Trent Gerew
		
		\today
	\end{FlushRight}
\end{document}